\section{Grafos}

\subsection{Min Cut Max Flow Duality}
We seek to construct a binary string $S$ of length $n$ that minimizes a total cost. The cost is defined as follows:
\begin{itemize}
    \item If $S_i=0$, a cost of $A_i$ is incurred.
    \item If $S_i=1$, a cost of $B_i$ is incurred.
    \item If $S_i=1$ and $S_j=0$, a penalty of $C_{i,j}$ is incurred.
\end{itemize}
The total cost is the sum of all such costs and penalties for the chosen string $S$.
\begin{enumerate}
    \item For each node $i$, add an edge $s \to i$ with capacity $B_i$. This represents the cost of setting $S_i=1$.
    \item For each node $i$, add an edge $i \to t$ with capacity $A_i$. This represents the cost of setting $S_i=0$.
    \item For each pair $(i, j)$ with a penalty, add an edge $i \to j$ with capacity $C_{i,j}$. This represents the penalty for setting $S_i=1$ and $S_j=0$.
\end{enumerate}

The capacity of a cut in this graph corresponds to the total cost of the binary string defined by the partition. For example, if node $i$ is in the $T$-partition ($S_i=1$) and node $j$ is in the $S$-partition ($S_j=0$), the edge $i \to j$ must be cut, adding the penalty $C_{i,j}$ to the total cost. By the max-flow min-cut theorem, the minimum cost is equal to the maximum flow from $s$ to $t$.


\subsection{Notable Applications and Equivalences on Flow}

\begin{itemize}
    \item \textbf{Bipartite Matching:}\\
    The size of the maximum matching in a bipartite graph is equal to the maximum flow in a network constructed from the graph.
    
    \item \textbf{Kőnig's Theorem:}\\
    In any bipartite graph, the number of edges in the maximum matching is equal to the number of vertices in the minimum vertex cover.
    $$ \text{Maximum Matching} = \text{Minimum Vertex Cover} $$
    $$ \text{Maximum Independent Set} = |V| - \text{Maximum Matching} $$

    \item \textbf{Menger's Theorem:}\\
    The maximum number of vertex-disjoint paths between two vertices $u, v$ is equal to the minimum number of vertices to be removed to disconnect $u$ and $v$.
        
    \item \textbf{Project Selection Problem (Min-Cut):}\\
    Binary decision problems with interdependent costs and profits can be modeled as a minimum cut problem, where the cut separates the "chosen" decisions from the "not chosen" ones.
\end{itemize}