\section{Combinatória}

\subsection{Stars and Bars Bounded}

De quantas formas podemos definir uma sequencia $x$ tal que:

$$ \sum^{n}_{i=0} x_i = M $$
$$ l_i \leq x_i \leq r_i $$

\subsubsection{Existência de solução}

A existência  de solução é garantida se e somente se:

$$ \sum^{n}_{i=0} l_i \leq M \leq \sum^{n}_{i=0} r_i $$

\subsubsection{Solução com DP em $ O(NM^2) $}

$$ DP(N, M) = \sum^{r_n}_{i=l_n} DP(N-1, M-i) $$

$$
DP(0, M) =
    \begin{cases}
        0, M \neq 0 \\
        1, M = 0
    \end{cases}
$$

\subsubsection{Caso Particular $ l_i = 0 , r_i = M$}

Stars and Bars com N-1 Barras

$$ Resposta = \binom{M+N-1}{M} $$

\subsubsection{Redução do problema para $l_i = 0$}

$$ M \xleftarrow{} M - \sum^{n}_{i=0} l_i $$
$$ x_i \xleftarrow{} x_i - l_i $$
$$ r_i \xleftarrow{} r_i - l_i $$
$$ l_i \xleftarrow{} 0 $$

\subsubsection{Caso Especial $ r_i = R $}

$$ Resposta = \sum^{N}_{k=0} (-1)^k \binom{N}{k} \binom{M - (R + 1) k + N - 1}{N-1} $$

\subsection{Formulas Básicas}

\subsection*{Permutação Circular}
O número de maneiras de ordenar $n$ objetos distintos em um círculo.
$$ PC_n = (n-1)! $$

\subsection*{Permutação com Repetição}
O número de maneiras de ordenar $n$ objetos, onde existem $n_1$ objetos idênticos de um tipo, $n_2$ de outro, e assim por diante.
$$ P_n^{n_1, n_2, \dots, n_k} = \frac{n!}{n_1! n_2! \dots n_k!} $$

\subsection*{Combinação Caótica (Desarranjo)}
O número de permutações de $n$ objetos onde nenhum objeto permanece em sua posição original.
$$ D_n = n! \sum_{k=0}^{n} \frac{(-1)^k}{k!} $$
Aproximação para $n$ grande:
$$ D_n \approx \frac{n!}{e} $$

\section*{Generalization for calculating number of elements in exactly \( r \) sets}
A princípio da inclusão-exclusão pode ser reescrito para calcular o número de elementos que estão presentes em zero conjuntos:
\[
\left| \bigcap_{i=1}^{n} A_i \right| = \sum_{m=0}^{n} (-1)^m \sum_{\substack{|X|=m}} \left| \bigcap_{i \in X} A_i \right|
\]
Considere sua generalização para calcular o número de elementos que estão presentes em exatamente \( r \) conjuntos:
\[
\left| \bigcup_{|B|=r} \left[ \bigcap_{i \in B} A_i \cap \bigcap_{j \notin B} A_j \right] \right| = \sum_{m=r}^{n} (-1)^{m-r} \binom{m}{r} \sum_{\substack{|X|=m}} \left| \bigcap_{i \in X} A_i \right|
\]