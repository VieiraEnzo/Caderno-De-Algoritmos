\section{Geometry}
    
\subsection{Pythagorean Triples}

For all natural $a, b, c$ satisfying $a^2 + b^2 = c^2$ there exist $m, n \in \mathbb{N}$ and $m > n$ such that (reverse is also true):
$$a = m^2 - n^2 \qquad b = 2mn \qquad c = m^2 + n^2$$

\subsection{Heron's Formula}

The area of a triangle can be written as $A = \sqrt{s\,(s-a)\,(s-b)\,(s-c)}$, where $a, b, c$ are the lengths of its sides and $s = \frac{a+b+c}{2}$.

This can be generalized to compute the area $A$ of a quadrilateral with sides $a, b, c, d$, with $s = \frac{a+b+c+d}{2}$ and $\alpha, \gamma$ any two opposite angles:

$$ A = \sqrt{(s-a)(s-b)(s-c)(s-d) - abcd\left( \cos ^2 \left( \frac{\alpha+\gamma}{2} \right) \right)} $$

\subsection{Pick's Theorem}

The area of a simple polygon whose vertices have integer coordinates is:

\[
A = I + \frac{B}{2} - 1
\]

where $I$ is the number of interior integer points, and $B$ is the number of integer points in the border of the polygon.

\subsection{Colinear Points}

Three points are colinear on $\mathbb{R}^2$ iff:

$$ \begin{vmatrix}
x_A & y_A & 1 \\
x_B & y_B & 1 \\
x_C & y_C & 1 \\
\end{vmatrix}  = 0 $$

The absolute value of this determinant is twice the area of the triangle $ABC$.

\subsection{Coplanar Points}

Four points are coplanar in $\mathbb{R}^3$ iff:

$$ \begin{vmatrix}
x_A & y_A & z_A & 1 \\
x_B & y_B & z_B & 1 \\
x_C & y_C & z_C & 1 \\
x_D & y_D & z_D & 1 \\
\end{vmatrix}  = 0 $$

\subsection{Trigonometry}

\subsubsection{Angle Sum}

$$ \sin(a \pm b) = \sin a \cos b \pm \cos \ \sin b $$
$$ \cos(a \pm b) = \cos a \cos b \mp \sin a \sin b $$
$$ \tan (a \pm b) = \frac{\tan a \pm \tan b}{1 \mp \tan a \tan b}$$

\subsubsection{Sum-to-Product Transformation}

$$ \sin a \pm \sin b = 2 \sin\frac{a \pm b}{2} \cos\frac{a \mp b}{2} $$
$$ \cos a + \cos b = 2 \cos\frac{a+b}{2} \cos\frac{a-b}{2} $$
$$ \cos a - \cos b = -2 \sin\frac{a+b}{2} \sin\frac{a-b}{2} $$
$$ \tan a \pm \tan b = \frac{\sin(a \pm b)}{\cos a \cos b} $$

\subsection{Centroid of a polygon}

The coordites of the centroid of a non-self-intersecting closed polygon is:

$$ \frac{1}{3A} \left(\sum_{i = 0}^{n-1}(x_i+x_{i+1})(x_iy_{i+1} - x_{i+1}y_i) , \sum_{i = 0}^{n-1}(y_i+y_{i+1})(x_iy_{i+1} - x_{i+1}y_i) \right), $$

where $A$ is twice the signed area of the polygon.

\subsection{2D Shapes}

\subsubsection{Square}
\begin{itemize}
    \item \textbf{Perimeter:} $P = 4s$
    \item \textbf{Area:} $A = s^2$
\end{itemize}
Where $s$ is the side length.

\subsubsection{Rectangle}
\begin{itemize}
    \item \textbf{Perimeter:} $P = 2(l + w)$
    \item \textbf{Area:} $A = l \cdot w$
\end{itemize}
Where $l$ is the length and $w$ is the width.

\subsubsection{Triangle}
\begin{itemize}
    \item \textbf{Perimeter:} $P = a + b + c$
    \item \textbf{Area:} $A = \frac{1}{2} b \cdot h$
    \item \textbf{Heron's Formula (Area):} $A = \sqrt{s(s-a)(s-b)(s-c)}$, where $s = \frac{a+b+c}{2}$.
\end{itemize}
Where $a, b, c$ are the side lengths, $b$ is the base, and $h$ is the height.

\subsubsection{Circle}
\begin{itemize}
    \item \textbf{Circumference:} $C = 2\pi r = \pi d$
    \item \textbf{Area:} $A = \pi r^2$
\end{itemize}
Where $r$ is the radius and $d$ is the diameter.

\subsubsection{Parallelogram}
\begin{itemize}
    \item \textbf{Perimeter:} $P = 2(a+b)$
    \item \textbf{Area:} $A = b \cdot h$
\end{itemize}
Where $a, b$ are adjacent side lengths, $b$ is the base, and $h$ is the height.

\subsubsection{Trapezoid}
\begin{itemize}
    \item \textbf{Area:} $A = \frac{1}{2}(a+b)h$
\end{itemize}
Where $a$ and $b$ are the parallel side lengths and $h$ is the height.


\subsection{3D Shapes}

\subsubsection{Cube}
\begin{itemize}
    \item \textbf{Surface Area:} $SA = 6s^2$
    \item \textbf{Volume:} $V = s^3$
\end{itemize}
Where $s$ is the side length.

\subsubsection{Rectangular Prism (Cuboid)}
\begin{itemize}
    \item \textbf{Surface Area:} $SA = 2(lw + lh + wh)$
    \item \textbf{Volume:} $V = lwh$
\end{itemize}
Where $l, w, h$ are the length, width, and height.

\subsubsection{Sphere}
\begin{itemize}
    \item \textbf{Surface Area:} $SA = 4\pi r^2$
    \item \textbf{Volume:} $V = \frac{4}{3}\pi r^3$
\end{itemize}
Where $r$ is the radius.

\subsubsection{Cylinder}
\begin{itemize}
    \item \textbf{Lateral Surface Area:} $A_L = 2\pi rh$
    \item \textbf{Total Surface Area:} $SA = 2\pi rh + 2\pi r^2 = 2\pi r(h+r)$
    \item \textbf{Volume:} $V = \pi r^2 h$
\end{itemize}
Where $r$ is the radius and $h$ is the height.

\subsubsection{Cone}
\begin{itemize}
    \item \textbf{Lateral Surface Area:} $A_L = \pi r l$
    \item \textbf{Total Surface Area:} $SA = \pi r l + \pi r^2 = \pi r(l+r)$
    \item \textbf{Volume:} $V = \frac{1}{3}\pi r^2 h$
\end{itemize}
Where $r$ is the radius, $h$ is the height, and $l = \sqrt{r^2+h^2}$ is the slant height.

\subsubsection{Pyramid}
\begin{itemize}
    \item \textbf{Volume:} $V = \frac{1}{3} A_b \cdot h$
\end{itemize}
Where $A_b$ is the area of the base and $h$ is the height.